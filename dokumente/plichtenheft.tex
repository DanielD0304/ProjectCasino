\documentclass[a4paper,12pt]{article}
\usepackage[utf8]{inputenc}
\usepackage[ngerman]{babel}
\usepackage{geometry}
\geometry{a4paper, left=25mm, right=25mm, top=25mm, bottom=25mm}

\title{Pflichtenheft}
\author{Ihr Name}
\date{\today}

\begin{document}

\maketitle

\tableofcontents

\section{Einleitung}
\subsection{Zweck des Dokuments}
Beschreiben Sie hier den Zweck des Pflichtenhefts.

\subsection{Zielbestimmung}
\subsubsection{Musskriterien}
Listen Sie hier die Musskriterien auf.

\subsubsection{Wunschkriterien}
Listen Sie hier die Wunschkriterien auf.

\subsubsection{Abgrenzungskriterien}
Beschreiben Sie hier die Abgrenzungskriterien.

\section{Produktumgebung}
\subsection{Software}
Beschreiben Sie hier die Softwareumgebung.

\subsection{Hardware}
Beschreiben Sie hier die Hardwareumgebung.

\subsection{Schnittstellen}
Beschreiben Sie hier die Schnittstellen.

\section{Produktfunktionen}
Beschreiben Sie hier die Funktionen des Produkts.

\section{Produktdaten}
Beschreiben Sie hier die Daten des Produkts.

\section{Leistungsanforderungen}
Beschreiben Sie hier die Leistungsanforderungen.

\section{Benutzeroberfläche}
Beschreiben Sie hier die Benutzeroberfläche.

\section{Qualitätsanforderungen}
Beschreiben Sie hier die Qualitätsanforderungen.

\section{Annahmebedingungen}
Beschreiben Sie hier die Annahmebedingungen.

\section{Glossar}
Definieren Sie hier wichtige Begriffe.

\end{document}